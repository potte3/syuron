\documentclass[:11pt, a4paper, titlepage]{jsarticle}
% 数式
\usepackage{amsmath,amsfonts,amssymb}
\usepackage{bm}
% 画像
\usepackage[dvipdfmx]{graphicx}
\usepackage{here}
\graphicspath{{./figure/}}
% 余白
\usepackage[top=20truemm,bottom=20truemm,left=20truemm,right=20truemm]{geometry}

\title{口頭試問原稿}
\author{福島晟}
\date{\today}

\begin{document}
\maketitle

稲垣研の福島晟です。発表を始めます。よろしくお願いします。
僕の卒業研究のテーマは「ヘリオトロンJプラズマにおける突発的現象の統計解析」です。
この「ヘリオトロンJ」とは京都大学が所有する核融合プラズマ実験装置です。

核融合発電の実用化においてプラズマの不安定性が大きな障害となっています。
磁場閉じ込めプラズマは高温高圧であるため様々な不安定性を持ちます。そして、一旦不安定化するとわずかな揺らぎが短時間で成長し、プラズマが大きく変形、変動してしまうため閉じ込めが難しくなります。
このような不安定性の中に「突発的現象」と呼ばれる磁場の揺らぎが成長する速度が速く、いつ発生するか予測が難しい現象があります。
突発的現象にはプラズマの持っている巨大なエネルギーを短時間で放出するものもあり、装置の破壊を引き起こす危険性があります。
この研究ではその突発的現象の統計的性質を解析し、発生のメカニズムを明らかにすることを目的としました。

ヘリオトロンJにはプラズマの電磁揺らぎ、磁場変動を計測する磁気プローブという計測装置が多数設置されており、解析ではその信号を用いました。
まず、磁気プローブ信号から励起状態の信号成分をバンドパスフィルタを通して取り出し、その信号に対し包絡線検波を行いました。
拡大した包絡線波形を見ると小さなバーストが複数回起きてそれがクラスター化し振幅が増大する現象のように見えます。そこでこの一つ一つのバーストのピークを調べることで突発的現象の解析を行うことにしました。
まず、バースト現象が起きている励起状態の部分と定常状態の部分に分けています。基準としてはこの式を用いました。その結果ショット72351での励起状態閾値電圧は0.1027Vとなりました。
また、電圧ごとのピークの分布は右側の図のようになりました。

イベントがランダムかつ独立に発生するときはガンマ分布に従うことを利用してイベントのピーク間隔のガンマ分布への当てはめを行いました。パラメータは最小二乗法によって推定しました。
閾値電圧を0.025V以上と低くして全体のイベントについて考えた時に比べて、0.1Vと高くして大きなイベントについてのみ考えた時は残差二乗和が4倍ほどになっていて、図からも高くした時は適合しておらず、低くした時は適合していると言えます。
すなわち,全体のイベントについて考えるとランダムかつ独立にイベントが発生していますが、大きなイベントについてのみ考えた時はイベント間に何らかの関係が存在していることが考えられます。
そこでHawkes過程というモデルを仮定しました。

バーストがクラスター化するモデルの一つにHawkes過程という点過程モデルがあります。
これは「不規則な時間間隔で発生する事象の発生間隔をモデル化した確率過程」で地震における余震の発生間隔、株式市場、Twitterの拡散等に適用されます。
Hawkes過程には自己励起性があり、外からのショックがなくても比較的短い時間内に複数の事象が集中して発生する現象が起きうるようになります

実際にショット72351に対してHawkes過程への推定を行いました。
左側が閾値電圧を0.025Vと低くした時の推定結果で、右側が閾値電圧を0.1027Vと高くした時の推定結果です。
まず、包絡線に対して閾値電圧を超えるピーク時刻を取得し点過程データとし、その点過程データを用いてパラメータを勾配降下法によって最尤推定しました。
閾値電圧を低くした時は大きなバーストを再現することができていませんが、高くした時は再現することができています。
また、強度関数の積分値を見ると閾値電圧を低くした時は直線に近い形となり、発生確率が一定なポアソン分布に近い状態となっています。
逆に、閾値電圧を高くした時は階段状の形となり、バーストのクラスター化が起きていることがわかります。

Hawkes過程を用いるとイベントがクラスタリングする様子を再現することができました。すなわち、大きなイベントが発生すると、しばらくの間、再び大きなイベントが発生する確率が高くなることが示唆されました。
今後は、突発的現象の他の統計的性質を調べることで発生メカニズムを明らかにして、最終的には予測を可能にすることを目指しています。
\end{document}
%  [0.01022064 0.91464623 0.69350145]
% [0.00541344 0.9978651  0.41287602]
