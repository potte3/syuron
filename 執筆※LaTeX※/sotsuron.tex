\documentclass[sotsuron]{kuee}
\usepackage{amsmath}
% 画像
\usepackage[dvipdfmx]{graphicx}
\usepackage{here}
\graphicspath{{./figure/}}
% 改ページしない
\usepackage{etoolbox}
\patchcmd{\chapter}{\cleardoublepage}{}{}{}
\patchcmd{\chapter}{\clearpage}{}{}{}

\title{ヘリオトロンJ における突発現象の時空間構造の観測}
\etitle{Observation of the Spatiotemporal Structure of abrupt Phenomena in HeliotronJ }
\author{大西 國憲}
\eauthor{Kuniaki Onishi}
\professor{稲垣 滋 教授}
\date{令和7年2月10日}

\begin{document}
\maketitle
\begin{eabstract}

  In order to realize fusion reactor, it is necessary to confine plasma for a long time.
  High-temperature, high-pressure plasma have very strong non-equilibrium and non-linearity, 
  resulting in abrupt phenomena like the sun,
  in which fluctuations of plasma grow so fast that it is difficult to predict when they will occur.
  This is very dangerous because some of such phenomena released the large energy of plasma in short time, 
  which may destroy the equipment installed inside the reactor vessel.
  Therefore, it is important to understand features of these abrupt phenomena.
  Here I report on the identification of the spatio-temporal structure 
  of abrupt bursts of magnetic field fluctuations observed in HeliotronJ .
  In abrupt burst events, the amplitude of the magnetic field fluctuations 
  in the plasma at 30-50 kHz suddenly increased and decreased in 1-2 ms. 
  it was identified that the wave-like fluctuation at 30-50 kHz (carrier wave) 
  has a mode structure of m = 6 or 9 and n = 4, where m (n) is the poloidal (toroidal) mode number. 
  the envelope of the carrier wave was found to have 
  mode structure of  \( m = n = 0 \), distinct from the carrier wave. 
  In addition, the edge electron temperature fluctuates synchronously with the envelope, 
  suggesting that the burst events were driving the transport. 

\end{eabstract}
\clearpage
\tableofcontents

%====================================================================================================================
%====================================================================================================================
%
% 序論
%
%====================================================================================================================
%====================================================================================================================
\chapter{序論}
\section{プラズマと核融合発電}
日本の一次エネルギー供給の多くは化石燃料による火力発電に依存しているが,
これには地球温暖化や酸性雨の原因となる二酸化炭素や窒素酸化物の排出という問題がある.
また,日本のエネルギー自給率は$6\%$程度と低く\cite{エネルギー自給率},輸入依存が高いため安定供給が困難である.
原子力発電は核分裂反応を利用しているが,放射性廃棄物の処理や事故リスクが問題視されている.
再生可能エネルギーは供給の不安定性や発電コストの高さが課題である.
これらの問題を解決する手段として,二酸化炭素を排出せず,燃料として豊富な水素を利用する核融合発電が期待されている.
核融合発電は,軽い原子核同士が衝突して重い原子核を作る際のエネルギーを利用する発電方式である\cite{プラズマ物理入門}.
この反応では,原子核の質量欠損$\Delta m$が生じ,アインシュタインの関係式
\begin{equation}
  E = \Delta mc^2
\end{equation}
に従ってエネルギーが放出される.核融合を起こすには原子核同士のクーロン力による反発を乗り越える必要があり,
そのためには$10^8$K以上の高温プラズマを生成・維持する必要がある.
現在の核融合研究では,重水素$D$と三重水素$T$の反応が最も有望とされている.代表的な核融合反応として以下が挙げられる:
\begin{align}
  D + T &\rightarrow {}^4He + n + 17.58\text{MeV} \label{eq: fusion_1} \\
  D + {}^3He &\rightarrow {}^4He + p + 18.34\text{MeV} \label{eq: fusion_2} \\
  D + D &\rightarrow T + p + 4.04\text{MeV} \label{eq: fusion_3} \\
  D + D &\rightarrow {}^3He + n + 3.27\text{MeV} \label{eq: fusion_4}
\end{align}
これらのうち,式(\ref{eq: fusion_1})は最も起こりやすく,約$1$億度の高温が必要とされる.
核融合反応は核分裂と異なり,基本的に長寿命の放射性廃棄物を生成しない.
ただし,中性子による放射化の影響で発電施設の廃棄時には慎重な処理が求められる.
燃料である重水素は海水中に豊富に存在し,三重水素もリチウム$Li$を利用して生成可能である.
また,式(\ref{eq: fusion_2})の$D$-$^3He$反応は中性子を伴わないため,最も理想的な反応とされるが,
実現には$10^9$K以上の温度が必要であり,技術的な課題がある.
核融合発電は温室効果ガスを排出しないため,エネルギー問題と環境問題を同時に解決できる可能性がある.
しかし,プラズマの高温維持が課題であり,実現には磁場を利用したプラズマの閉じ込め技術が不可欠である.
磁場閉じ込め装置には主にトカマク方式とヘリカル方式があり,本研究ではヘリカル方式の一種であるヘリオトロンJを用いた.
ヘリカル方式は定常的なプラズマ閉じ込めが可能であり,核融合炉の実現に向けた重要な研究対象である.
詳細については第2章で述べる.

%===============================================================================================================

\section{磁場閉じ込めプラズマにおける突発的現象}
核融合発電の実用化における大きな問題の一つとしてプラズマの不安定性がある.
磁場閉じ込めプラズマは高温高圧であるため様々な不安定性が存在する.
そして,一旦不安定化するとわずかな揺らぎが成長し,プラズマは大きく変形,変動してしまうため閉じ込めることができなくなる.
したがって,プラズマを高温・高密度に長時間閉じ込めておくためには,この不安定性を抑えることが重要である.
このような不安定性の中に「突発的現象」と呼ばれる磁場の揺らぎが成長する速度が速く,いつ発生するか予測が困難な現象が存在する\cite{プラズマの突発現象と遠非平衡性}.
核融合プラズマでは非常に大きなプラズマの蓄積エネルギーとそれを閉じ込めるための磁場エネルギーが存在する.
突発的現象には,プラズマの持っているほとんどのエネルギーを短時間で放出するのもあり装置を破壊してしまう可能性があり大変危険である.
突発的現象について理解は進んできているが,不安定性を駆動する自由エネルギーは十分あるにも関わらず,発生のタイミングが一定でないなど不明な点が多い.
プラズマ以外にも太陽のプロミネンス,火山の噴火,雪崩など様々な突発的現象があり,これらの突発的現象を広く扱えるようなモデル,予測方法が求められている.

%===============================================================================================================

\section{本研究の目的}
核融合プラズマ実験装置であるヘリオトロンJでは突発的現象が頻繁に観測されている.
核融合発電を実現するためにはこの突発的現象を理解することが重要である.
したがって,突発的現象の統計的性質を解析することで突発的現象の発生機構を明らかにすることが本研究の目的である.
\clearpage

%====================================================================================================================
%====================================================================================================================
%
% 実験・計測装置
%
%====================================================================================================================
%====================================================================================================================
\chapter{ヘリオトロンJ 装置} \label{sec:Helio}
今回の実験で使用した実験装置の概要と使用した計測装置の原理について説明する.
本研究で使用したのは京都大学エネルギー理工学研究所で稼働しているヘリオトロンJ装置である\cite{HeliotronJ実験}.
この装置は中型の先進ヘリカル系プラズマ実験装置であり,ヘリカル軸ヘリオトロン配位の実機実験を通じた最適化を目的としている.
主なパラメーターとしては主半径R=1.2m,平均小半径a=0.1-0.2m,中心磁場B\textless1.5Tである.
ヘリオトロンJ装置の磁場コイルシステムは図\ref{fig: Heliotron J model}のように(曲数L)/(ピッチ数M)=1/4連続巻ヘリカルコイル(HFC),
強弱二種類のトロイダルコイル(TFC-A,TFC-B),及び3組のポロイダル・コイル(VFC,IVFC,AVFC)から構成されている.
HFC,TFC-A,TFC-B,VFC,IVFC,AVFCのそれぞれに独立制御可能な電源を置くことで,広範囲な磁場配位制御が可能となっている.

%===============================================================================================================

\section{ヘリカル方式}
ヘリオトロンJでプラズマを閉じ込めるにあたり用いられているヘリカル方式について他の代表的な磁場閉じ込め方式であるトカマク方式と比較しつつ説明する.
磁場閉じ込め方式では電荷を持つ粒子が磁力線に沿って,その周りをサイクロトロン運動により回転する.
したがって,磁力線の端と端を繋げて円環を作り端をなくすとプラズマを長時間保持することができる.
実際にプラズマを閉じ込めるにはトーラス磁場と呼ばれる単純なドーナツ状の磁場にひねること,すなわち回転変換を加える必要がある.
単純なトーラス磁場では曲率半径方向に磁場の値が変わっているため,
サイクロトロン運動の回転半径は曲率半径方向位置によって値が変わり,トーラスの上下方向にドリフト運動する.
このとき,プラズマを構成するイオンと電子では回転方向が逆のため,それぞれ反対方向にドリフトする.
これによって,ドーナツ状に生成したプラズマの上下の表面にイオンあるいは電子が現れ,荷電分離が発生する.
この荷電分離による電界$\textbf{E}$によって,プラズマは$\textbf{E}\times\textbf{B}$ドリフト運動を起こし,
トーラス磁場の曲率半径方向,すなわちプラズマのドーナツが広がる不方向に飛散するためトーラス磁場に閉じ込められない.
回転変換を与えるとドーナツの上下の磁力線がつながるため,
主に電子が磁力線に沿って運動することにより荷電分離した電荷が中和され,
電界$\textbf{E}$が発生しなくなることからプラズマを閉じ込めることができるようになる.

回転変換の与え方にはトカマク方式とヘリカル方式の二種類がある.
トカマク方式は単純なトーラス磁場に生成したプラズマにプラズマ電流を流して,この電流で発生した磁場によってひねる.
原理的にトランスを利用してプラズマ電流を流すためプラズマ保持時間は400秒,あるいは1000秒程度と定常的ではなくなる.
トカマク方式の利点としては装置がヘリカル方式より簡単で,短時間であれば高温高密度プラズマを容易に生成することができる点が挙げられる.
ヘリカル方式ではトーラス磁場を形成する電磁石をはじめからひねっておくことで回転変換を与えている.
プラズマの閉じ込めに必要なすべての磁場が電磁石で形成されることにより,プラズマを定常的に生成保持することができる.
したがって,保守期間以外は常に稼働することが求められている発電所での利用に適している\cite{核融合炉への道程と大型ヘリカル装置}.

磁力線がトーラス方向に1周する間にポロイダル方向に$\iota\,rad$だけずれるとすると回転変換は$\iota/2\pi$と表される。
回転変換はポロイダルモード数$m$,トロイダルモード$n$数を用いて
\begin{equation}
  \frac{\iota}{2\pi} = \frac{n}{m}
\end{equation}
と表すことが出来る。回転変換の値は,プラズマの安定性や波動の伝搬特性に密接に関係し,特定の条件下ではモードの共鳴が発生する可能性がある。
\ref{ch:result}章で用いたヘリオトロンJ の磁場配位では、$\iota/2\pi \approx 0.55$ であり、
この値が,観測された揺らぎのモード構造に影響を与えていると考えられる。

\begin{figure}
  \centering
  \includegraphics[width=60mm]{chapter3/heliotronJ.png}
  \caption{Heliotron J装置概要図}
  \label{fig: Heliotron J model}
\end{figure}

%===============================================================================================================

\section{加熱装置}
ヘリオトロンJ装置ではプラズマの生成と加熱システムとして電子サイクロトロン加熱(ECH),中性粒子ビーム入射(NBI)が用いられている.
ECHはプラズマ中に電子サイクロトロン周波数,またはその高調波に近い周波数の波をプラズマに入射し,
電子の旋回運動に対して共鳴相互作用を起こすことで電子にエネルギーを与え加熱する方法である.
ヘリオトロンJでは通常,70GHz第二高調波X-mode ECHが用いられている.ここで異常モード(Xモード)とは
電磁波の電場が閉じ込め磁場の磁力線に対して垂直方向に振動しながら伝搬するモードである.
NBIは中性の水素粒子に高エネルギーを与え加速させ,プラズマに入射させる.
電離で生じた高エネルギー水素イオンは電子と衝突し,電子にエネルギーを与え加熱する.
今回の実験ではECHによる加熱が用いられておりNBIは使用されていない\cite{HeliotronJ実験}.

%===============================================================================================================

\section{計測装置}
ヘリオトロンJでは非常に高温なプラズマが閉じ込められているため接触型センサーを使うことができない.
そのため非接触計測が必要となる.以下に各種センサーの概要を述べる.
電子密度を計測するために用いられるのがマイクロ波干渉計である.
プラズマの誘電率は電子密度に依存するため電磁波の伝搬速度はプラズマ中では真空中よりも速くなる.
この性質を利用して,ヘリオトロンJでは130GHzのマイクロ波を用いてプラズマ中を伝搬した電磁波と
真空中を伝搬した電磁波を干渉させ,その位相差を測定することで電子密度を求めている.
電子温度計測には電子サイクロトロン放射計測装置(\ref{sec:ECE}節参照)やトムソン散乱計測装置が用いられている.
YAGレーザーをプラズマ中に入射すると,その電界により電磁波が新たに放射される.
これがトムソン散乱である.放射された電磁波の波長スペクトルから電子の速度分布関数がわかり,
この速度分布関数をマクスウェル分布と仮定することで電子温度を求めている.
また可視光の分光器やフィルターモノクロメーターによって原子,分子からの発光を観測している.
水素(重水素)は電磁にH$\alpha$を放出する.この発光強度を測定することでプラズマへの粒子供給がわかる.
また,他にもプラズマ内の不純物イオンによる線スペクトル放射があり,
可視領域の不純物発光線を測定することで不純物の流入をモニターすることができる.
蓄積エネルギー計測のために反磁性コイルも用いられる.
トロイダルプラズマではプラズマ圧力勾配により反磁性電流が流れる.
この電流が作る磁束密度の変化を反磁性コイルによって検出する.
反磁性コイルはプラズマ断面を取り囲むように配置しているので,
その信号強度はプラズマ全体の反磁性電流の空間積分となり,
プラズマの圧力すなわちプラズマの蓄積エネルギーに比例するため計測できる.
また,プラズマの電磁揺らぎの計測のためにヘリオトロンJには磁気プローブが多数設置されている.
本研究ではこの磁気プローブと電子サイクロトロン放射計測装置を主に用いたので次節で詳しく説明する.

%===============================================================================================================

\subsection{磁気プローブ} \label{sec:mp}
磁気プローブはミルノフコイルと呼ばれ,小さなソレノイドコイルである.
静電気的揺らぎはプラズマの外まで伝搬しないが,プラズマ電流の揺らぎによって生じる電磁揺らぎはプラズマの外まで伝搬するためプラズマの外側に配置した磁気プローブで検出が可能である.
磁気プローブはファラデーの電磁誘導の法則に基づいて時間変動する磁場および電場を計測する装置である.ファラデーの法則は
\begin{equation}
  \nabla\times\textbf{E} = -\frac{\partial\textbf{B}}{\partial t}
  \label{eq: Faraday}
\end{equation}
で表される.コイル断面積上での全磁場変動を考えると,ストークスの定理より,誘導電場のコイルの銅線に沿った線積分はコイル両端における電圧$V_s$は
\begin{equation}
  V_s = \int \nabla\times\textbf{E} ds = -NS\frac{\partial\textbf{B}}{\partial t}
  \label{eq: V_s}
\end{equation}
で現れる.ここで,$\textbf{E}$は電場,$\textbf{B}$はコイルと鎖交する磁束密度,$N$はコイルの巻数,$S$はコイルの断面積である.
式(\ref{eq: V_s})からわかるように計測する電圧は磁束密度の時間微分(時間変化)であり,計測結果を磁場強度に戻すには時間積分する必要があるが,本研究では揺らぎ発生の検出に使用するため磁場強度に戻すことはしない.

磁気プローブは時間微分計測であるため周波数応答に優れている.
しかし,コイルには有限なキャパシタンスが存在するため,測定回路はLCR回路となり共振が発生する.
共振周波数は$\omega =1/\sqrt{LC}$である.
ここで,$L$はコイルのインダクタンス,$C$は回路の全容量である.
帯域を広げるには共振周波数を上げることが重要である.
キャパシタンス$C$は磁気プローブのケーブルの長さを短くすると小さくなる.
インダクタンスLは$L\propto N^2 lS$であるため,コイルの巻数$N$,コイル断面積$S$,
コイルの高さ$l$を減らす必要があるが,磁気プローブで計測する磁場が部分的なことを考慮する必要がある.
また,キャパシタンスと並列に抵抗などを接続することで共振を抑えることもできる.
ヘリオトロンJではトロイダル方向に4個,ポロイダル方向に14個の磁気プローブが真空容器壁面上に設置されている.
トロイダル方向の4個の磁気プローブ(Troidal Magnetic Probe : TMP)は
同一ポロイダル断面形状となる4箇所の各トロイダルセクションにほぼ90度ずつ離れるように設置されている.
また,各磁気プローブはポロイダル,トロイダルそして径方向の磁場揺動の3方向成分を測定できるような構成となっている(図\ref{fig: MP_CG}).
ポロイダル方向の14個の磁気プローブ(Poloidal Magnetic Probe : PMP)は
図\ref{fig: MP_place}のように同一ポロイダル断面状に10度または20度間隔で設置されていて,全体で180度の方向をカバーできるようになっている.
磁気プローブは500kHz程度までの高周波磁場揺動が測定可能であり,通常のサンプリング周波数は1MHzである.

\begin{figure}
  \centering
  \includegraphics[width=60mm]{chapter3/MP_CG.png}
  \caption{磁気プローブ(トロイダルアレイ)の外観図(CG)}
  \label{fig: MP_CG}
\end{figure}
\begin{figure}
  \centering
  \includegraphics[width=60mm]{chapter3/MP_place.png}
  \caption{磁気プローブ(ポロイダルアレイ)の設置位置}
  \label{fig: MP_place}
\end{figure}

%===============================================================================================================

\subsection{電子サイクロトロン放射計測} \label{sec:ECE}
電子サイクロトロン放射(ECE:Electron Cyclotron Emission)とは,
プラズマ中の高温電子が磁場中でサイクロトロン運動(渦旋運動)を行う際,
その運動エネルギーの一部をマイクロ波領域の電磁波として放出する現象である.
この放射の周波数は,局所の磁場強度に依存して決まり,また放射強度は局所の電子温度に比例するため,
各位置で放出されるECEを検出することにより,プラズマ内部の電子温度の時間変動を把握することが可能となる.

ヘリオトロンJ ではラジオメータによりECE計測をしている。
高帯域のECEを16chフィルターバンクによりダウンコンバートしそれぞれ2乗検波した後、
ECE強度に変換し、1MHzの高速ディジタライザでデジタル化している。

本装置では,複数の観測チャネルのうち特に以下の2つの高速観測チャネルが用いられている.
ch2はECE周波数$59-60$~GHz,観測位置はプラズマの外縁部に相当する規格化平均小半径($\rho$)が約$0.95$である.
プラズマエッジ近傍の電子温度分布や揺動を高時間分解能で追跡するために利用され,
エッジ領域はプラズマ閉じ込め性能や外部との相互作用に大きく影響するため,ここでの計測は重要である.
ch8はECE周波数$70-72$~GHz,観測位置はプラズマ中心部に近い規格化平均小半径($\rho$)が約$0.09$である.
プラズマ中心部は最も高温な領域であり,その温度分布や揺動情報は,プラズマ内部のエネルギー輸送や閉じ込め状態の評価に直結する.
ch8は,中心部の急激な温度変動を詳細に捉えるために使用される.
以降ではch2,ch8によって計測される温度をそれぞれ$Te(0.95)$, $Te(0.09)$と表現する.

このように,電子サイクロトロン放射計測装置は,
プラズマ内部での局所的な電子温度を正確に把握するための強力な診断手段となっている.
特に,異なる周波数帯($59-60$~GHzおよび$70-72$~GHz)に対応したch2とch8を用いることで,
プラズマのエッジ部と中心部それぞれの温度状態や揺動を高精度かつ高時間分解能で測定でき,
核融合実験におけるプラズマの制御・安定化に寄与する重要な情報を提供している.
\clearpage

%====================================================================================================================
%====================================================================================================================
%
% 解析手法
%
%====================================================================================================================
%====================================================================================================================
\chapter{解析手法}
\section{スペクトル解析}
本研究では突発現象の時空間構造を同定する.この時空間構造の特徴づけにスペクトル解析を用いる.
スペクトル解析とは時間$\rightarrow$周波数,空間$\rightarrow$波数と領域(domain)を変換し,揺らぎのパワーの周波数と波数に対する依存性を示したものである.
突発現象を成分に分解することで,突発現象の全体構造を理解しやすくなる.

解析では時系列信号を周波数領域で表すためにフーリエ変換を用いた. 
実際に得られるのは有限かつ離散的な信号であるため, 離散フーリエ変換を用いて解析した. 
離散的な $N$ 個のデータ $x(n) (n = 0, 1, 2, \dots, N)$ が与えられたとき, 
この離散フーリエ変換 $X(k)$ は
\begin{equation}
  X(k) = \frac{T}{N} \sum_{n=0}^{N-1} x(n) \exp \left( -i 2\pi \frac{jk}{N} \right) \quad (k = 0, 1, 2, \dots, N/2)
\end{equation}
と表される. ここで $T$ は解析区間である. 
また時間分解能 $\Delta t = T/N$, 周波数分解能 $\Delta f = 1/T$, 
ナイキスト周波数 $f_N = N/(2T)$ である. 
離散フーリエ変換は取り出した有限区間が無限に繰り返されることを仮定しているので, 
不連続点による誤差が生じる. この誤差を小さくするために離散フーリエ変換を行う前に窓関数としてハン窓を乗じている. 

各周波数のエネルギーの大きさを表したパワースペクトル $P(f)$ は
\begin{equation}
P(f) = \left\langle \frac{1}{T} X(f) X^*(f) \right\rangle = \left\langle \frac{1}{T} \left|X(f)\right|^2 \right\rangle
\end{equation}
となる. ここで $\langle \dots \rangle$ はアンサンブル平均, $^*$ は複素共役を表している. \cite{スペクトル解析}
アンサンブル平均をすることによってノイズの影響を低減している. 
しかし周波数分解能とノイズの影響の大きさはトレードオフの関係にあり, 
解析では特徴的な周波数を特定しやすいように適当なアンサンブル平均数を設定している. 

%===============================================================================================================

\subsection{クロスパワースペクトル}
パワースペクトルは単一の磁気プローブの信号から求めることができるが, モードを同定するには, 波動の伝搬を検出しなければならない. 
伝搬を検出するには複数のプローブによる同時多点計測が必要になる. 波動の伝搬は振動の位相差から推定することができる. 
その位相差の推定にクロスパワースペクトルを用いる. 
クロスパワースペクトル密度は, 2つの離散時間信号を掛け合わせたパワースペクトル密度である. 
次の節で説明するコヒーレンスとフェイズを算出するために利用する. 
クロスパワースペクトル密度は複素数関数で, その位相は二つの信号間の位相差を表す. 
二つの時系列信号データx(t)とy(t)があるとした時, そのフーリエ変換を$X(f),Y(f)$としてクロスパワースペクトル$S_{XY}(f)$は
\begin{equation}
  S_{XY}(f) = X(f) Y^*(f)
\end{equation}
と表される. $Y^*(f)$は$Y(f)$の複素共役を表す. 
$X(f)=a_{X}e^{i(2πf+θ_{X})}, Y(f)=a_{Y}e^{i(2πf+θ_{Y})}$とすれば$S_{XY}(f) = a_{X}a_{Y}e^{i(θ_{X}-θ_{Y})}$であり,振幅と位相差の情報を持つ. 

%===============================================================================================================

\subsection{コヒーレンスと位相差} \label{sec:coh}
二つの信号の周波数空間での相関を解析する手法としてコヒーレンスとフェイズがある. 
信号$x$,$y$間における二乗コヒーレンス関数$\gamma^2 (f)$は,$x$のパワースペクトル密度が $P_{xx}$, $y$ の
パワースペクトル密度が $P_{yy}$, $x$, $y$ のクロスパワースペクトル密度が $P_{xy}$ であるとすると, 
\begin{equation}
  \gamma^2 (f) = \frac{\langle |S_{xy} |^2 \rangle}{\langle S_{xx} \rangle \langle S_{yy} \rangle}
\end{equation}
で表される. コヒーレンスとは規格化したパワースペクトルであり, 範囲は$0 \leq \gamma(f) \leq 1$ となる. 
コヒーレンスが高いほど二つの信号は相関が高い事を示す. 
コヒーレンスを正確に求めるにはアンサンブル平均が重要である. 
アンサンブル数が $1$ の場合, 定義からコヒーレンスは $1$ である. 
アンサンブル数が少ないと, 偶発的に相似な雑音があった場合, コヒーレンスが高く出てしまう. 
コヒーレンスでは significance level より十分に大きいことが重要である. 
$\gamma$ の significance level は$\approx \frac{1}{\sqrt{N}}$であり, アンサンブル数の平方根の逆数である. 
$N=1$ では significance level も $1$ となり有意に相関が強いとは言えない. 
\ref{ch:result}章の解析においては6ms(データ点数6000点), 周波数分解能1.8kHzとし、
アンサンブル数は $N=20$ 、この時significance level $\approx0.22$となる。
コヒーレンスが高いとは位相関係が揃っている事を意味する. 
この時, 位相差が意味を持つ. 
この信号間の位相差を表すのがクロスフェイズ $\theta_{xy} (f)$ である. 
$S_{XY} (f) = a_X a_Y e^{i(\theta_X - \theta_Y )}$ であるため, クロスフェイズは, 
\begin{equation}
  \theta_{xy} (f) = \tan^{-1} \left( \frac{\operatorname{Im}(S_{xy})}{\operatorname{Re}(S_{xy})} \right)
\end{equation}
で表される. 
$\operatorname{Re}(S_{xy})$, $\operatorname{Im}(S_{xy})$は$S_{xy}$の実部と虚部である. 
$-\pi \leq \theta_{xy} (f) \leq \pi$で定義した. 
このため, $\theta_{xy}$ には $2k\pi$ ($k=0,1,2,\dots$)の不確定性がある. 
実際の位相差は時間変動や伝搬の向きを考慮して $2k\pi$ を復元しなければならない. 
これを位相アンラッピングと呼ぶ. 
実際の実験ではポロイダルおよびトロイダル方向に配置したプローブの信号から波の位相差を調べた. 
まずノイズの影響をあまり受けない強いコヒーレンスを持つ成分を抽出する必要がある. 
次に, 抽出した成分のクロスフェイズを導出し, 位相アンラッピングを施して伝搬を観測し, 
その伝搬を最もよく説明できるポロイダルモード数$m$, トロイダルモード数$n$を求める. 

%===============================================================================================================

\section{ヒルベルト変換による包絡線検波} \label{sec:hilbert}
信号の包絡線を調べることで特定の周波数の振幅の成長率や立ち上がり時間などの物理的に意味のある情報を抽出することができる.
本研究ではヒルベルト変換によって信号の包絡線を求めた.\cite{ヒルベルト変換}
ある周波数の連続時間信号の一般式は,その包絡線を$A(t)$ とすると
\begin{equation}
  x(t) = A(t)\cos \omega t
\end{equation}
と表すことができる.ここで元の時間信号の位相を $\pi/2$ ずらした複素信号を足すと
\begin{equation}
  z(t) = x(t) + i y(t) = A(t)\cos \omega t + i A(t)\sin \omega t = A(t)e^{i \omega t}
\end{equation}
となる.こうして得られた信号を解析信号と呼ぶ.解析信号の絶対値をとることで瞬時振幅すなわち包絡線を得る.
\begin{equation}
  A(t) = |z(t)| = \sqrt{x(t)^2 + y(t)^2}
\end{equation}
ここで元の時間信号の位相を $\pi/2$ ずらした複素信号は次の式で得られる.
\begin{equation}
  i y(t) = \mathcal{F}^{-1}[\operatorname{sgn}(f) \cdot \mathcal{F}[t]]
\end{equation}
ここで $\operatorname{sgn}(f)$ は符号関数で関数の中身が負なら $-1$,正なら $1$ をとる関数である.
またこの変換をヒルベルト変換と呼ぶ.このとき周波数領域で負の周波数成分の正負を反転させることが時間領域で位相を
$\pi/2$ ずらす操作に等しいという性質を用いている.
実際のプログラムでは以下の式のように元信号のフーリエ変換に単位ステップ関数$u(f)$
を掛けて逆フーリエ変換することで解析信号を求めている.
\begin{align}
  z(t) &= \mathcal{F}^{-1}[\mathcal{F}[t]] + \mathcal{F}^{-1}[\operatorname{sgn}(f) \cdot \mathcal{F}[t]] \\
  &= \mathcal{F}^{-1}[\mathcal{F}[t] + \operatorname{sgn}(f) \cdot \mathcal{F}[t]]\\
  &= \mathcal{F}^{-1}[2 u(f) \mathcal{F}[t]]
\end{align}.
\clearpage

%====================================================================================================================
%====================================================================================================================
%
% 時空間構造の解析
%
%====================================================================================================================
%====================================================================================================================
\chapter{実験結果と解析} \label{ch:result}
\section{典型的な突発現象}
今回の研究で解析する対象の実験はヘリオトロンJの標準的磁場配位,磁気軸における磁場強度1.25Tで行った.
ヘリオトロンJの磁場配位は各種コイルに流される電流によって形成される.
本実験では(HFC, TFC-A, TFC-B, AVFC, IVFC)=(86\%,79\%,44\%,74\%,71\%)で行われた.
それぞれヘリカルコイルー主垂直磁場コイル電流, Aトロイダルコイル電流, Bトロイダルコイル電流, 付加垂直磁場コイル電流, 
内側垂直磁場コイル電流を意味し,定格電流に対する割合で示されている.
プラズマの生成は70GHzのミリ波による電子サイクロトロン共鳴加熱によって行われている.

典型的なプラズマパラメータの時間変化を図\ref{fig: params@72351}に示す.
Gas Puffは真空容器内に注入する重水素ガスの流量を表している.
平均電子密度は$1\times10^{19}m^{-3}$程度であり,ヘリオトロンJのECHプラズマとして典型的な放電である.
プラズマは220ms-320msの間で安定的に生成されているが,磁気プローブ信号(MP)が270msで突然強くなり285msで元の強度に戻っている.
これは磁気揺らぎが突発的に励起されているということを示している.
この突発的現象の直前,直後の平均電子密度(ne)はほとんど変化していないため,突発的現象はプラズマの密度変化によって引き起こされているとは考えにくい.
また,燃料ガスである水素の電離,すなわち電子源の大きさの指標であるH$\alpha$光,不純物の流入の指標である可視波長領域(Visible)における発光強度,
及びAXUV(Absolute eXtream Ultra Violet)素子による極真空紫外領域の発光強度の変化も小さい.
磁気揺らぎを励起すると不安定性には閾値を持つものが多いが,
以上の観測結果からこの突発的現象は閾値を持つような不安定性によって引き起こされたものではないと考えられる.
そこで,この突発的現象は確率的に引き起こされるものだと考え,その統計性を磁気プローブ信号について解析することで明らかにする.
電子サイクロトロン放射(Electron Cyclotron Emission:ECE)計測の信号強度に関しては\ref{sec:peak}節で詳しく検証している. 
周辺部のプラズマ温度が突発現象の発生に合わせて変動しており(詳細は\ref{sec:peak}節参照), 熱輸送に影響を与えている可能性がある. 

\begin{figure}
  \includegraphics[width=130mm]{chapter4/params@72351.png}
  \caption{代表的なプラズマパラメータの時間発展}
  \label{fig: params@72351}
\end{figure}

%===============================================================================================================

\subsection{突発現象の周波数構造}
図\ref{fig:MP4_wave}はプラズマ内の磁束密度の変化を積分し時系列の電圧信号に変換したものである.
横軸は時間(ms),縦軸は電圧(V)である.
この図より220ms-265msの間では電圧が安定であるが,280ms付近で突然電圧が強くなり励起され,290ms付近で元の電圧に戻っている.
この突発現象の周波数構造を調べるため, スペクトル解析を行なった. 
250ms$\sim$256msと278ms$\sim$284msのそれぞれ6ms(6000点に相当)の時間窓に対して
FFTによりパワースペクトル密度を求め, その時間窓をスライドさせスペクトルの時間発展を計算した. 
結果を図\ref{fig:MP4_PSD_mer}に示す. 横軸が時間, 縦軸が揺らぎ周波数であり, 赤色が濃いほどパワースペクトル密度が高い事を示す. 
この図より, 突発現象発生時刻とした時刻278ms$\sim$284msにのみ強いピークを持つことが分かる. 

\begin{figure} % MP4波形
  \includegraphics[width=130mm]{chapter4/MP4_wave.png}
  \caption{突発現象発生時の磁気プローブの波形}
  \label{fig:MP4_wave}
\end{figure}
\begin{figure} % PSD
  \includegraphics[width=130mm]{chapter4/MP4_PSD_mer.png}
  \caption{突発現象発生時のパワースペクトル密度(左が拡大図)}
  \label{fig:MP4_PSD_mer}
\end{figure}

%===============================================================================================================

\subsection{包絡線検波} \label{sec:env_detect}

安定な定常状態であると考えられる250ms-256msと,突発現象の発生が確認された278ms-284msのそれぞれ6msの区間について
フーリエ変換を用いてどのような周波数成分が含まれているかを調べた.
その結果が図\ref{fig:MP4_spe}である.
横軸は周波数(kHz),縦軸はパワースペクトル密度(V$^2$/kHz)であり,縦軸は対数スケールである.
青い線は250ms-256msの区間のパワースペクトル密度であり,赤い線は278ms-284msの区間のパワースペクトル密度である.
図\ref{fig:MP4_spe}より,突発現象発生時(278ms-284ms)は定常状態(250ms-256ms)に比べて
周波数30kHz-50kHzの帯域のパワースペクトル密度が大きいことがわかる.

この帯域の揺らぎ成分の振幅が突発的に変動すると考えられるため,30kHz$\sim$50kHzの帯域の信号を取り出すために
バンドパスフィルタを通して,\ref{sec:hilbert}節で述べたヒルベルト変換を用いた包絡線検波を行った.
その結果が図\ref{fig:MP4_env_wave}である.横軸は時間(ms),縦軸は電圧(V)である.
黒の波形が元信号をバンドパスフィルタに通した信号であり,赤の波形がバンドパスフィルタを通した信号の包絡線である.
この図をおよそ278ms-284msについて拡大したものが図\ref{fig:MP4_env_wave_zoom}である.
図\ref{fig:MP4_env_wave},\ref{fig:MP4_env_wave_zoom}をみると,突発的現象は包絡線の急変動(包絡線バースト)が複数回起きて,
それがクラスター化している現象だと考えられる.
包絡線に対して, その元となる30kHz$\sim$50kHzの波動をキャリアと呼ぶ. 
このキャリアの時空間構造はどのような不安定性が磁気的揺らぎを駆動しているかを同定するための重要な情報である. 
同様に, バーストの時空間構造を明らかにすることは突発現象の背後に潜む物理仮定の同定に有効である. 

\begin{figure} % MP4スペクトル解析
  \includegraphics[width=130mm]{chapter4/MP4_spe.png}
  \caption{突発的現象が起こった時と起こっていない時のパワースペクトル密度}
  \label{fig:MP4_spe}
\end{figure}
\begin{figure} % MP4包絡線
  \includegraphics[width=130mm]{chapter4/MP4_env_wave.png}
  \caption{バンドパスフィルタを通した磁気プローブ波形の包絡線}
  \label{fig:MP4_env_wave}
\end{figure}
\begin{figure} %MP4包絡線拡大図
  \includegraphics[width=130mm]{chapter4/MP4_env_wave_zoom.png}
  \caption{バンドパスフィルタを通した磁気プローブ波形の包絡線(拡大)}
  \label{fig:MP4_env_wave_zoom}
\end{figure}

%===============================================================================================================

\section{モード数推定手法} \label{sec:mode_method}
波動の時空間構造とは主に周波数と波数によって表される. 
トロイダル面上を伝搬可能な波動の波数はポロイダルモード数とトロイダルモード数によって表される. 
本研究では磁気プローブ信号の位相差を解析することでモード数を推定した. 具体的な手法を以下に示す. 
プラズマの波形を$\cos{(m\theta + n\phi + \omega t)}$とするとプラズマの位相は$m\theta + n\phi + \omega t$となる. 
2つの磁気プローブで観測されたプラズマの位相差を$\Delta$とすると
\begin{equation}
  \Delta = n\Delta \phi + m\Delta \theta
\end{equation}
となる. ポロイダル磁気プローブのch1$\sim$ch14を考えるとき, 
\ref{sec:mp}節に示すように各磁気プローブの設置位置のトロイダル角は一定であるため$\Delta \phi = 0$となる. 
この時,位相差は下式で表される. 
\begin{equation}
  \Delta = m\Delta \theta
\end{equation}
これはPMP(ch1)$\sim$PMP(ch14)で観測されたプラズマの位相差と磁気プローブの設置角が比例関係にあることを示している. 
よって理想的な信号に対して縦軸が位相差,横軸が磁気プローブの設置角となるようなグラフを作成すると傾きがモード数となるような直線と重なることがわかる. 
本研究では, 求めた位相差を横軸がプローブの設置角, 縦軸が位相差となるグラフにプロットし, 
各点がどの直線と最も接近するかを調べることでモード数の推定を行った. 
この際\ref{sec:coh}節で述べた$360k (k=0, \pm 1, \pm 2, \pm, \dots)$の任意性を考慮して位相アンラッピングを行い, 
各モード数に対応する直線に最も接近するように$360k$を復元した. 

どの直線と最も接近するのかを判別する方法としては以下で定義される残差の二乗平均の加算平均を取った値を用いた. 
\begin{equation}
  r = \frac{1}{N} \sum_{i=1}^{N} \sqrt{(y_i - \hat{y}_i)^2}
\end{equation}
ここで$N$はデータ点の総数, $y_i$ は観測された位相差, $\hat{y}_i$ は理論的な直線モデルによる予測値である. 
$r$は, データが理論モデルにどれだけ適合しているかを数値的に評価するために用いられる. 
本研究では, 異なるモード数に対して$r$を計算し, 最小の$r$を示すモード数を最良推定値として選定した. 

また, 本研究で使用したトロイダル磁気プローブ(TMP)は, わずかにポロイダル方向にずれがあるため, 
トロイダルモード数を正しく推定するにはこのずれを考慮する必要がある. 
位相差ずれ$\delta$は, ポロイダル方向のモード数$m$を使って, 
\begin{equation}
  \delta = m\Delta\theta \label{eq:hosei}
\end{equation}
と表される. よってトロイダルモード数を推定するためには
まず初めにポロイダルモード数を推定した後にポロイダル角のずれを補正しなければならない. 


%===============================================================================================================

\section{キャリアのモード数の推定} \label{sec:car}
キャリアのモード数の推定結果を示す. \ref{sec:mode_method}節で述べた通り, 
TMP(ch1) $\sim$ TMP(ch4)にはわずかなポロイダル方向のずれが存在する. 
これを補正するためにポロイダルモード数$m$が必要となるため, まずはPMPに対して解析を行った. 
図\ref{fig:PMP(all)_wave_zoom}に示すようにPMP信号はノイズ成分が大きく, 
解析の精度を向上させるためには不要な信号をあらかじめ取り除いておく必要がある. 
不要なPMP信号を選定する方法としては, TMP(ch1)とのコヒーレンスを求め, 
30kHz-50kHzにおいて値が0.8一定以下となる信号を取り除いた. 
TMP(ch1)に対するPMP(ch1) $\sim$ PMP(ch14)とのコヒーレンスを図\ref{fig:coh_PMPtoMP1_merged}に示す. 
右図は30-50kHz付近で左図を拡大したものである. 
この図の破線は下から順に, コヒーレンスの値が0.7, 0.8, 0.9となるラインを示している. 
30kHz-50kHzにおいてコヒーレンスの最大値が0.8以下となる
PMP(ch8), PMP(ch9), PMP(ch10), PMP(ch11), PMP(ch12), PMP(ch14)は除外する. 
残った信号のうちPMP(ch5), PMP(ch6), PMP(ch7)に関して, コヒーレンスは0.8を超えているが
他の5つの信号に比べて平均的に値が低いため, 
ここではPMP(ch5), PMP(ch6), PMP(ch7)を含めるパターンAと含めないパターンBの2通りで解析を行った. 
パターンA, パターンBのPMP信号を用いてPMP(ch1)を基準としたコヒーレンスを求めたところ
図\ref{fig:PMP_coh_mer(AB)}のようになった. 
上段はパターンAの位相差とコヒーレンスを表しており, 右図は左図を拡大したものである. 
同様に下段はパターンBの位相差とコヒーレンスを示しており, 右図は左図を拡大した図である. 
赤色のバツ印で示された点は30kHz $\sim$ 50kHzにおいてコヒーレンスが0.8を超えるような点であり, 
これが解析の対象となる. 
2つのパターンそれぞれに対して位相差を計算し, 位相アンラッピングを用いて値を復号したのちに残差二乗平均を取った. 
結果を図\ref{fig:PMP_resi}に示す. 
これより残差二乗平均値は$m=6\,\mathrm{or}\,9$において最小となることがわかる. 
よってポロイダルモード数は$m=6\,\mathrm{or}\,9$と推定される. 

次にトロイダルモード数$n$を推定する. 
ポロイダルモード数は$m=6\,\mathrm{or}\,9$と推定されたため, 
式(\ref{eq:hosei})を用いて2パターンで位相差の補正を行った. 
図\ref{fig:MP_coh_mer}にTMP(ch1)を基準としたTMP(ch1)$\sim$TMP(ch4)のコヒーレンスを示す. 
赤色のバツ印で示された点は30kHz$\sim$50kHzにおいてコヒーレンスが0.8を超える点である. 
このような点に対して位相差を計算してプロットしたグラフと残差二乗平均の推移を図\ref{fig:MP_resi}に示す. 
赤点は$m=9$として位相差を補正した結果で, 青点は$m=6$として位相差を補正した結果となる. 
よって残差の二乗平均は$n=4$において最小となることがわかる. 
ゆえにトロイダルモード数は$n=4$と推定される. 

最終的にキャリアのモード数は, トロイダルモード数が$n=4$ポロイダルモード数が$m=6\,\mathrm{or}\,9$と推定された. 
各モード数に対してフィッティングした位相差をプロットしたグラフを図\ref{fig:PMP_phasediff_mer}, 
図\ref{fig:MP_phasediff}に示す. 
これは位相アンラッピング後の位相差の大きさが$360$を超えた際に
値を$\pm 360k$を加えることで, 位相差を表す縦軸を$-180$から$180$の範囲に来るように調節している. 
この結果を見ると確かに$n=4$\, $m=6\,\mathrm{or}\,9$において直線に接近していることがわかる. 

本解析ではトロイダルモード数$n = 4$,ポロイダルモード数$m = 6$\,or\,$9$が得られた.
このモードの励起機構について考察する.
閉じた磁気面上を伝搬する波動のモード構造は,その磁気面の磁力線の回転変換と密接な関係があり,
\begin{equation}
    \frac{\iota}{2\pi} = \frac{n}{m}
\end{equation}
を満たす揺らぎが共鳴的に励起される可能性が高い. 
今回のヘリオトロン J 実験で用いた磁場配位は,$\iota/2\pi\approx 0.55$ である.
$m=6$の時$n/m \approx 0.66$, $m=9$の時$n/m \approx 0.44$
であり, プラズマ電流等の影響により回転変換分布が変われば, どちらのモードも共鳴条件を満たす可能性がある. 

\begin{figure} % PMP波形
  \includegraphics[width=130mm]{chapter4/env/PMP/PMP(all)_wave_zoom.png}
  \caption{PMP(ch1)$\sim$PMP(ch14)の波形(ピーク付近で拡大)}
  \label{fig:PMP(all)_wave_zoom}
\end{figure}

\begin{figure} % PMPコヒーレンス(MP1基準)
  \includegraphics[width=130mm]{chapter4/coh_PMPtoMP1_merged.png}
  \caption{TMP(ch1)に対するコヒーレンスと位相差(左は拡大図)}
  \label{fig:coh_PMPtoMP1_merged}
\end{figure}

\begin{figure} % PMPコヒーレンス(パターンA, B)
  \includegraphics[width=130mm]{chapter4/car/PMP/PMP_coh_mer(AB).png}
  \caption{パターンA,BのPMP信号のコヒーレンス}
  \label{fig:PMP_coh_mer(AB)}
\end{figure}

\begin{figure} % PMP残差
  \includegraphics[width=130mm]{chapter4/car/PMP/PMP_resi.png}
  \caption{PMP信号の残差二乗平均}
  \label{fig:PMP_resi}
\end{figure}

%トロイダル
\begin{figure} % MPコヒーレンス
  \includegraphics[width=130mm]{chapter4/car/MP/MP_coh_mer.png}
  \caption{MP信号のコヒーレンス}
  \label{fig:MP_coh_mer}
\end{figure}

\begin{figure} % MP残差
  \includegraphics[width=130mm]{chapter4/car/MP/MP_resi.png}
  \caption{MP信号の残差二乗平均}
  \label{fig:MP_resi}
\end{figure}

%まとめ
\begin{figure} % PMP位相差プロット
  \includegraphics[width=130mm]{chapter4/car/PMP/PMP(AB)_m69_phasediff_mer.png}
  \caption{PMPの位相差(上図A,下図B)}
  \label{fig:PMP_phasediff_mer}
\end{figure}

\begin{figure} % MP位相差プロット
  \includegraphics[width=130mm]{chapter4/car/MP/MP_phasediff.png}
  \caption{MPの位相差(赤点:$m=9$, 青点:$m=6$)}
  \label{fig:MP_phasediff}
\end{figure}



%===============================================================================================================

\section{包絡線のモード数の推定} \label{sec:env}
\ref{sec:car}節と同様にして包絡線のモード数を推定した. 
MPのポロイダル方向のずれを考慮するためにPMPの解析を先に行った. 
また, ノイズの影響を考慮するためにPMP(ch1)$\sim$PMP(ch14) をパターンA,Bに分けて解析を行った. 
\ref{sec:env_detect}節で述べたように突発現象は 30kHz$\sim$50kHz においてPSDのピークをもつ。
そこで図\ref{fig:PMP(all)_wave_zoom}に示すPMP信号に対して
30kHz$\sim$50kHzのパンドパスフィルタをかけ, 包絡線を求めた. 
結果を図\ref{fig:PMP(all)_env_wave}に示す. 
これらの包絡線に対して位相差とコヒーレンスを計算した結果を図\ref{fig:PMP(AB)_coh}に示す. 
上段はパターンAの位相差とコヒーレンスを表しており, 右図は左図を拡大したものである. 
同様に下段はパターンBの位相差とコヒーレンスを示しており, 右の図は左の図を拡大した図である. 
赤色のバツ印で示された点は0kHz$\sim$15kHzにおいてコヒーレンスが$0.8$を超える点である. 
ここから位相差を計算し, 位相アンラッピングを用いて値を復号したのちに残差二乗平均を取った. 
結果を図\ref{fig:PMP_env_resi}に示す. 残差二乗平均値を確認すると$m = 0$において最小となることがわかる. 
よってポロイダルモード数は$m = 0$と推定される. 

次に包絡線のトロイダルモード数を推定する. 
ポロイダルモード数が$m=0$と推定されたため, 式(\ref{eq:hosei})の補正項は値がゼロとなる. 
PMPと同様にして TMP(ch1) $\sim$ TMP(ch4) に対しても
30kHz $\sim$ 50kHzでバンドパスフィルタをかけて包絡線を求めた. 
結果を図\ref{fig:MP_env_wave}に示す. 
包絡線に対して位相差とコヒーレンスを計算した結果が図\ref{fig:MP_env_coh_mer}となり, 
赤色バツ印はコヒーレンスが0.8以上となる点を示す. 
コヒーレンスが0.8以上となるような点を対象に残差二乗平均を計算した結果を
図\ref{fig:MP_env_resi}に示す. 
よってトロイダルモード数は$n=0$と推定される. 

最終的に包絡線のモード数は, トロイダルモード数が$n=0$ポロイダルモード数が$m=0$と推定された. 
各モード数に対してフィッティングした位相差をプロットしたグラフを
図\ref{fig:MP_env_phasediff_mer}, 図\ref{fig:PMP_env_phasediff_mer}に示す. 
図\ref{fig:MP_env_phasediff_mer}の右図は左図の拡大図を表す. 
図\ref{fig:PMP_env_phasediff_mer}は左図, 右図はそれぞれパターンA, パターンBの
PMP信号の包絡線に対して位相差をプロットした図である. 
残差二乗平均の値からトロイダルモード数もポロイダルモード数もともに0と推定されたが, 
この図を見るとプロットした点は確かに横軸(傾き0の直線)に接近していることがわかる. 
これらの解析から包絡線はキャリアとは異なる時空間構造を持っていることが明らかとなった. 
プラズマ中で$m=n=0$の空間構造を持つ揺らぎにゾーナル流が考えられる. 
突発性を理解する上でキャリアとゾーナル構造を持つ揺らぎとの非線形結合が重要なことが示唆される. 

%ポロイダル
\begin{figure} %PMP包絡線波形
  \includegraphics[width=130mm]{chapter4/env/PMP/PMP(all)_env_wave_zoom.png}
  \caption{PMP信号(包絡線)}
  \label{fig:PMP(all)_env_wave}
\end{figure}

\begin{figure} %PMPコヒーレンス(A,B)
  \includegraphics[width=130mm]{chapter4/env/PMP/coh/PMP(AB)_env_coh.png}
  \caption{PMP信号の位相差とコヒーレンス}
  \label{fig:PMP(AB)_coh}
\end{figure}

\begin{figure} % PMP残差
  \includegraphics[width=130mm]{chapter4/env/PMP/PMP_env_resi.png}
  \caption{PMP信号(包絡線)の残差二乗平均}
  \label{fig:PMP_env_resi}
\end{figure}

%トロイダル
\begin{figure} % MP包絡線波形
  \includegraphics[width=130mm]{chapter4/env/MP/MP_env_wave_zoom.png}
  \caption{MP信号の包絡線の波形}
  \label{fig:MP_env_wave}
\end{figure}

\begin{figure} % MPコヒーレンス
  \includegraphics[width=130mm]{chapter4/env/MP/MP_env_coh_mer.png}
  \caption{MP信号の包絡線のコヒーレンス}
  \label{fig:MP_env_coh_mer}
\end{figure}

\begin{figure} % MP残差
  \includegraphics[width=130mm]{chapter4/env/MP/MP_env_resi.png}
  \caption{MP信号(包絡線)の残差二乗平均}
  \label{fig:MP_env_resi}
\end{figure}

\begin{figure} % MPプロット
  \includegraphics[width=130mm]{chapter4/env/MP/MP_env_phasediff_mer.png}
  \caption{MP信号(包絡線)のコヒーレンス(右が拡大図)}
  \label{fig:MP_env_phasediff_mer}
\end{figure}

\begin{figure} % PMPプロット
  \includegraphics[width=130mm]{chapter4/env/PMP/PMP_env_phasediff_mer.png}
  \caption{PMP信号(包絡線)の位相差グラフ(左図:パターンA,右図:パターンB)}
  \label{fig:PMP_env_phasediff_mer}
\end{figure}



%===============================================================================================================

\section{ピーク解析} \label{sec:peak}
本節ではヘリオトロンJ で観測された磁場揺らぎの突発現象が閉じ込め性能に与える影響を議論する. 
本解析では電子サイクロトロン放射(ECE)計測システムを用いて電子温度変化を測定し, 
プラズマ中で発生する突発現象が電子温度や輸送特性に与える影響を明らかにした. 

\ref{sec:ECE}節で述べたように, ECE計測システムは異なる位置に複数のチャネルを有している. 
ここでは$Te(0.95)$と$Te(0.09)$を用いて解析を行い, 
これらの比較を通じて突発現象による電子温度変化の時間応答を評価した. 

磁気プローブの包絡線がピークを取る時刻を突発現象が発生した時刻として定義し, 
突発現象発生時刻の前後10msの $Te(0.95)$, $Te(0.09)$ を解析対象とした. 
ショット72352におけるピーク時刻($t=273.575ms$)前後の$Te(0.95)$の波形を図\ref{fig:ECE@72352_mer}に示す. 
これはノイズ成分が強く, 真のECE信号の挙動を正確に理解することが困難である. 
%解析手法
複数のプラズマショットに対してピーク前後10msのデータを取り出し, 多くのショットデータを重ね合わせることで統計的な精度を向上させた. 
この処理により, 個々のショットに含まれるランダムなノイズ成分を低減し, 
突発現象に伴うECEデータの平均的な挙動を明確にすることが可能となる. 
各ショットにおけるECE信号の電圧データを$V^{(i)}(t)$とし, $L$ショット分のデータを用いて
平均化したECE信号$\bar{V}(t)$を以下の式で定義する.   
\begin{equation}
  \bar{V}(t) = \frac{1}{L} \sum_{i=1}^{N} V^{(i)}(t)
\end{equation}
ここで, $V^{(i)}(t)$は$i$番目のショットにおける時間$t$でのECE信号の電圧を表す. 
ノイズ低減の効果を評価するため, 分散$\sigma^2(t)$を以下のように定義する.   
\begin{equation}
  \sigma^2(t) = \frac{1}{L} \sum_{i=1}^{L} \left( V^{(i)}(t) - \bar{V}(t) \right)^2
\end{equation}
$L$が増加すると, 個々のショットに特有のランダムノイズが平均化され, $\bar{V}(t)$がより真のECE信号変動を反映するものとなる. 
したがって, 複数のショットのデータを重ね合わせることで, 突発現象によるECE信号の変化をより正確に把握することが可能となる. 
今回は725351から72374までのショットを用いたため,$L=22$となる. 
%結果の考察
包絡線がピークを取る時刻は図のように分布している。何らかの統計的性質はないようである。
ノイズを平均化した$Te(0.95)$,$Te(0.09)$,TMP(ch3)の波形を図\ref{fig:3peak}に示す. 
ショット毎にピーク時間を$t=0$として前後10msを取得して加算した波形であるため, 横軸は相対的な時間である. 
周辺温度$Te(0.95)$はキャリア振幅の増大と同時に低下し始め, キャリア振幅が減少に転じると突然増加する. 
突然の温度増加はsawtoothのように中心部から熱が吐き出されている可能性を示す. 
中心温度$Te(0.09)$はほとんど変化しないか突発現象後にやや低下する傾向がある. 
電子温度の低下は, エネルギー損失やプラズマの閉じ込め性能の低下を示唆する. 
突発現象によってエネルギーの輸送が過渡的、突発的に促進された可能性が考えられる. 
\begin{figure} % ECE@72352 
  \includegraphics[width=130mm]{chapter4/peak/ECE@72352_mer.png}
  \caption{$Te(0.95)$計測データ}
  \label{fig:ECE@72352_mer}
\end{figure}

\begin{figure}
  \includegraphics[width=130mm]{chapter4/peak/3peak.png}
  \caption{ピーク付近のECE信号データ}
  \label{fig:3peak}
\end{figure}
\clearpage

%====================================================================================================================
%====================================================================================================================
%
% 結論
%
%====================================================================================================================
%====================================================================================================================
\chapter{結論}
プラズマにおける突発現象はプラズマの不安定性を理解するために非常に重要である. 
本研究では, 突発現象の時空間構造を明らかにしてプラズマの不安定性の理解を深めることを目的として関して以下の結論を得た. 
\begin{itemize}
  \item キャリアのモード数を$n=4$, $m=6$\,or\,$9$と同定した
  \item 包絡線バーストのモード数は$m=n=0$であり, キャリアとは異なることを明らかにした. 
  \item 突発現象と同期して周辺温度が変動することを観測した. 
\end{itemize}

本研究の意義として, プラズマ物理学の観点からは, 突発現象のモード構造と電子温度変動の関連を明確にし, 
プラズマ不安定性の理解を深める一助となる. 特に, ゾーナル流のような$m=n=0$の成分が突発現象と関連している可能性が示唆される点は, 
非線形波動相互作用やエネルギー輸送メカニズムの解明に重要な知見を提供する. また, 核融合研究の観点からは, 
突発現象によるエネルギー散逸の影響を定量的に捉えることで, 閉じ込め性能の向上や不安定性制御のための基礎的データを提供できる. 
これは, 将来的な安定な核融合プラズマの維持に向けた研究に寄与するものである. 
一方で, PMP信号のノイズが強く, ポロイダルモード数測定の精度に限界があったことが今後の課題として挙げられる. 
より高精度なモード解析のためには, 磁気プローブの信号処理技術の向上や, より多点計測による詳細な空間構造の解析が必要である. 
また, 電子温度変動の詳細な評価には, ECEデータのノイズ低減や時間分解能の向上が求められる. 
さらに, 突発現象の発生機構をより包括的に理解するためには, 理論・数値シミュレーションとの比較検討を進めることが重要である. 
今後の研究において, これらの点を発展させることで, プラズマ輸送と不安定性に関する理解がさらに進展し, 
核融合炉の実現に向けた知見を蓄積できると考えられる. 



% %======================================================================
% %		謝辞
% %======================================================================
\clearpage
\begin{acknowledgements}
  本論文の作成にあたり,多大なるご指導,ご鞭撻を賜りました京都大学エネルギー研究所の稲垣滋教授に深く感謝いたします.
  また,門信一郎准教授,金史良助教にはゼミにおいて多くの助言をしていただきました.心より感謝いたします.
  さらに,研究室の先輩方,同期の皆様のおかげで充実した研究生活を送ることができました.深く感謝いたします.
  最後に,私が研究に打ち込めるように経済的,精神的に支えてくれた家族に感謝いたします.
\end{acknowledgements}

% %======================================================================
% %		参考文献 
% %======================================================================
\clearpage
\bibliographystyle{kueethesis}
\begin{thebibliography}{99}
  \bibitem{エネルギー自給率} 経済産業省. 長期エネルギー需給見通し. (2015), p7.
  \bibitem{プラズマ物理入門} 宮本健朗. プラズマ物理入門. \textbf{1} (1997), p5.
  \bibitem{プラズマの突発現象と遠非平衡性} 伊藤公孝, et al. 21aAE-6 プラズマの突発現象と遠非平衡性. 日本物理学会講演概要集, \textbf{71.1} (一般社団法人 日本物理学会, 2016).
  \bibitem{スペクトル解析} 辻義之, 他. 講義:流体乱流研究から診たプラズマ乱流データの解析-2. 相関とスペクトル解析, Journal of Plasma Fusion Research, Vol. 85, (2009). p620-p630.
  \bibitem{ヒルベルト変換} ONOSOKKI. 基礎からの周波数分析 ヒルベルト変換と解析信号, https://www.onosokki.co.jp/HP-WK/eMM\_back/emm180.pdf, (2024/02/04).
  \bibitem{HeliotronJ実験} 長崎百伸, et al. プロジェクトレビュー Heliotron J 実験. プラズマ・核融合学会誌, \textbf{96.9} (2020) p450-479.
  \bibitem{核融合炉への道程と大型ヘリカル装置} 小森彰夫. 核融合炉への道程と大型ヘリカル装置. 応用物理 85.5 (2016) p389-p395.
\end{thebibliography}
% 著者 タイトル 出版年 ページ   

% %======================================================================
% %		付録
% %======================================================================
\clearpage
\appendix
\chapter{付録}
\end{document}
